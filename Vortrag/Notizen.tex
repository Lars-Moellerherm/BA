\documentclass[
  bibliography=totoc,     % Literatur im Inhaltsverzeichnis
  captions=tableheading,  % Tabellenüberschriften
  titlepage=firstiscover, % Titelseite ist Deckblatt
]{scrartcl}

% größere Variation von Dateinamen möglich
\usepackage{grffile}
\usepackage{multicol}
\usepackage{graphicx}
\usepackage{amssymb}
\usepackage{amsmath}
\usepackage{xparse}
\usepackage{braket}
\usepackage{units}
\usepackage[locale=DE,separate-uncertainty=true,per-mode=reciprocal,output-decimal-marker={,},]{siunitx}
\sisetup{
  round-mode          = places, % Rounds numbers
  round-precision     = 2, % to 2 places
}
\usepackage[section]{placeins}
\usepackage{pdflscape}
\usepackage{expl3}
%Komma als Dezimaltrenner in der mathe Umgebung, um in Umgebungen wie [0, 2] ein Leerzeichen nach dem Komma zu erhalten einfach eins setzen
\usepackage{icomma}
\usepackage{cancel}


% Verbesserungen am Schriftbild
\usepackage{microtype}

% Hyperlinks im Dokument
\usepackage[
  unicode,        % Unicode in PDF-Attributen erlauben
  pdfusetitle,    % Titel, Autoren und Datum als PDF-Attribute
  pdfcreator={},  % ┐ PDF-Attribute säubern
  pdfproducer={}, % ┘
]{hyperref}
% erweiterte Bookmarks im PDF
\usepackage{bookmark}

% Trennung von Wörtern mit Strichen
\usepackage[shortcuts]{extdash}

\begin{document}
  \begin{itemize}
    \item Folie 3:
      \begin{itemize}
        \item geladene Teilchenstrahlung sorgt für auch für Schauer Untergrund;
        \item Satelliten: Detektionsfläche ist begrenzt: CT Atmosphäre ist Szintillator
        \item Kosmische Magnetfelder: Magnetare EInheit Gauß!!!!!!!!!!!!!!!!!%$\num{e12}-\SI{e15}{\G}$ ; Galaxien $\num{e-6}-\SI{e-4}{\G}$
      \end{itemize}
    \item Folie 4:
      \begin{itemize}
        \item SNR´s als hauptquelle bestätigen durch entdeckung neuer SNR
        \item Pulsarwind Nebel: Pulsarwind = starker Sonnenwind; Nebel= Supernovaeüberreste die durch Pulsar zum rotieren gebracht wird
        \item haronisches spektrum durch untersuchen der Gammaemissionen untersuchen. Hadronen wechselwirken mit Interstellaren Wolken und erzeugen so gammas
        \item Gammas mit großer Rotverschiebung machen Paarerzeugung mit hintergrundstrahlung
        \item EBL: rotverschobene Sterne und die reemission von ISM -> mehr erfahren über Geschichte des Universums
        \item Neutrinos wären der Beweis für hadronische Beschleunigung p->pi->neutrino
        \item Objekte mit großer Fluss-Variabilität (GRB) könnten Lorentz-Invarianz Verletzung testen
        \item Dunkle Materie (Neutralino), extragalaktische Quellen (AGN)
        \item Was ist wichtig:
        \begin{itemize}
          \item Wollen Spektren untersuchen: geringer relativer Fehler
          \item untersuchung eines Spektrums, jedoch hauptsächlich bei hohen Energien
          \item Untersuchung von Sepktren
          \item Genauigkeit bei geringen Energien
          \item ?
          \item Untersuchung von Spektren
          \item Untersuchung eines Spektrums
          \item Finden eines Linien Spektrums bei bestimmter energie: Absoluter Fehler minimieren
        \end{itemize}
      \end{itemize}
    \item Folie 5:
        \begin{itemize}
          \item Schauer: Heitler Model: Bremsstrahlung und Paarerzeugung. Elektron verliert Hälfte nach $R=X_0 \ln2$ ; Photon wechselwirkt durch
                Paarerzeugung nach gleicher Länge und verteilt Energie gleichmäßig. $X_0$ charakteristische Weglänge
        \end{itemize}
    \item Folie 6:
      \begin{itemize}
        \item erstes Moment = x,y und r,$\phi$
        \item zweites Moment = L und w
        \item phi und psi nicht gleich falls Ellipse anderen winkel hat
        \item Intensität ist auch Hillas Parameter
        \item Teilchenseparation schon geschehen, daher nur gamma ereignisse in Daten
      \end{itemize}
    \item Folie 7:
      \begin{itemize}
        \item RF setzt sich aus möglichst unkorrelierten Entscheidungsbäumen zusammen.
        \item Jedes Blatt schätzt den Mittelwert der Wahrheiten der Datenpunkte in dem Blatt $c_m$
        \item Minimieren des Kriterium bis Abbruchbedingung erfüllt(max depth, minimale Blattgröße)
        \item Kriterium minimiert des absoluten Fehler
        \item Der genutzte CART kann nur binär trennen. Gibt jedoch auch andere wie C5.0

      \end{itemize}
    \item Folie 8:
      \begin{itemize}
        \item Bagging um zufällige unabhängige Datensätze zu erzeugen. Ziehen von N Datensätzen von der größe M
        \item Bootstrapping um Datensatz nicht zu verkleinern: Datenpunkte zufällig ersetzen.
        \item Attribute zufällig ziehen um Verzerrung zu verkleinern jedoch Varianz wird größer.
        \item Am Ende Mittelwertbildung über alle Bäume
        \item Kann kein Übertraining entstehen, durch Vergrößerung des Waldes
      \end{itemize}
    \item Folie 9:
      \begin{itemize}
        \item Jedes Teleskop, welches das Schauer sieht schätzt eigenständig die Energie.
        \item Benötigen jedoch nur eine Energie für jedes Schauer. Daher Mittelwert bilden.
        \item Nicht nur Arithmetisches Mittel
        \item Training eines RF der gleich die Energie des Schauers schätzt. Hat jedoch zusätzlich die gemittelte Schätzung des ersten Waldes -> Verschachtelte Methode
      \end{itemize}
    \item Folie 10:
      \begin{itemize}
        \item diffuse: kommen zum Beispiel von ww der Teilchenstrahlung mit ISM
        \item Überschätzen ist zu erkennen.
        \item Linien durch Korreliertheit der Entscheidungsbäume
        \item $R^2$-Score: 1 : perfekte Vorhersage; 0 : genauso gut wie einfache Mittelwertbildung; <0 : absolut unbrauchbares Modell
        \item Mit vorsicht zu genießen: Empfindlich gegenüber Trends(wie hier); gleiche Anzahl an Datenpunkten für vergleich (nicht gegeben)
        \item $0.5$ wir sind besser als der bloße Mittelwert
      \end{itemize}
    \item Folie 11:
      \begin{itemize}
        \item ersten Vier Attribute sind Eventspezifisch und nicht Teleskopspezifisch; In dieser Information steckt ja auch mehr über die gesamte
        Energie die in dem Schauer deponiert ist.
        \item Hillas Parameter sind durch den Fehlenden Parameter "Abstand zum Schauermittelpunkt" kein Indiz für größe des Schauers
        \item die skalierte Größe und die größe sind nah beieinander und haben durchaus ausreißer -> Parameter sind nicht unabhängig
      \end{itemize}
    \item Folie 12:
      \begin{itemize}
        \item verschiedene Mittelwerte austesten
      \end{itemize}
    \item Folie 13:
      \begin{itemize}
        \item bei geringen Energien ist der Median besser und der Mittelwert bringt keine Verbesserung.
        \item Ab dem TeV bereich gibt es kaum einen Unterschied zwischen Mittelwert und Median ab 30TeV sogar schlechter als ohne Schätzung
        \item Hauptsächlich uberschätzen, wie schon in der Migrationsmatrix zu sehen.
        \item Ähnliches Bild bei Auflösung
        \item stärkere Fluktuation (nicht geklärt warum)
        \item Mittelwert aber schlechter als beim bias
        \item Bei großen Energien keine Verbesserung aber auch keine Verschlechterung
      \end{itemize}
    \item Folie 14:
      \begin{itemize}
        \item zwei Events herausgesucht.
        \item einzelne Ausreißer  sorgen dafür, dass das arithmetische Mittel schlechter wird, der Median ist jedoch gut
        \item Wie kann man berücksichtigen das einige Ausreißer sind und wie kann man Ausreißer erkennen?...
      \end{itemize}
    \item Folie 15:
      \begin{itemize}
        \item ... Berücksichtigen durch gewichtete Mittelung!
        \item Mögliche Ursache, ist das das Teleskop nicht das ganze Schauer sieht oder nur weit weg ist -> wenig Intensität
        \item Zusehen hohe Intensität, geringer relativer Fehler
        \item Problem: 1) Intensität abhängig von Energie (geringe energie nie viel Intensität) 2) Fehlerverlauf schon durch Eigenschaften des relativen
              Fehlers bedingt.
        \item Versuch ist es trotzdem wert, suchen jedoch noch ein anderes Gewicht
      \end{itemize}
    \item Folie 16:
      \begin{itemize}
        \item Es gibt ja unterschiedliche Teleskope, da sie unterschiedlich sensitiv sind.
        \item nutzen geschätzte energie für einteilung in Sensitivitätsbereiche. -> Problem, da es nur schätzung ist und somit falsch einsortiert werden kann
      \end{itemize}
    \item Folie 17:
      \begin{itemize}
        \item Sensitivität verbessert sich deutlich bei geringen energien.
        \item Die Intensität verschlecht das Ergebnis jedoch.
        \item Bei großen Energien mach die Gewichtung kaum einen Unterschied (relativer Fehler)
      \end{itemize}
    \item Folie 18:
      \begin{itemize}
        \item nutzen zweiten Random Forest
      \end{itemize}
    \item Folie 19:
      \begin{itemize}
        \item deutlich weniger Datenpunkte (logisch)
        \item immer noch linien -> Korreliertheit
        \item $R^2$ ist angestiegen von 0.5 auf 0.75, jedoch keine genaue aussage über algortihmus
        \item liegt enger um die Diagonale
      \end{itemize}
    \item Folie 20:
      \begin{itemize}
        \item Viele Ausreißer -> viele Attribute kommen alle aus der Schätzung des ersten RF daher strark Korreliert
        \item Auch hier Spielt das SST ein deutlich größere Rolle als alle anderen
        \item Die meisten Attribute spielen garkeine Rolle mehr
        \item Schätzung am wichtigsten -> sehr früh im Baum damit es erste Trennung vornimmt und dann Attribute genutzt werden die wichtig für diesen
              Energiebereich sind
      \end{itemize}
    \item Folie 21:
      \begin{itemize}
        \item Verbesserung in allen Energiebereichen
        \item Bei Auflösung das peak bei 0.1 TeV verschwunden
        \item größte Verbesserung bei niedrigen Energien
        \item durch die frühe Energietrennung kann sich der eine zweig auch auf die vorhersage für geringe Energien konzentrieren, was vorher
              nicht Interessant war da minimieren des absoluten Fehlers
      \end{itemize}
    \item Folie 22:
      \begin{itemize}
        \item Andere Möglichkeit um dem Algorithmus eine minimierung der geringen Energien interessant zu machen, ist es den Zielbereich zu
              verkleinern (durch Trafo)
        \item Nutze den logarithmus als bijektive Transformation
        \item die $3$ ist willkürlich gewählt, Ziel war nur keine großen energien für sehr kleine x zu erhalten
      \end{itemize}
    \item Folie 23:
      \begin{itemize}
        \item Wir betrachten die Eventspezifische Schätzung
        \item Das der Algortihmus kein Interesse an niedrigen Energien hat wird deutlich durch den CUT der hier zusehen ist.
        \item Cut sieht durch logarithmisierte Skala drastisch aus, für den Algorithmus bedeutet das jedoch kein großen Performanceverlust wenn er alle
              Energien <0.1TeV auf 0.1TeV schätzt.
        \item Die Transformation hat jedoch dazu geführt, dass er später abschneidet.
        \item jedoch ist der $R^2$ score abgesunken. Kein Wunder er wird schlechter für große Energien die mehr wiegen beim $R^2$Score
        \item muss sich jedoch genauer angeguckt werden.
      \end{itemize}
    \item Folie 24:
      \begin{itemize}
        \item Deutlich besser bei geringen Energien jedoch leicht schlechter bei hohen Energien
        \item aber immer Besser als bloße Mittelwertbildung
        \item Auflösung geht die gleichen Schwankungen wie ohne Trafo durch. -> es ist wirklich nur eine Trafo
        \item Wir sind bei Verzerrung fast die ganze zeit unter 1 und bei der Auflösung fast die ganze zeit unter 0.5
      \end{itemize}
    \item Folie 25:
      \begin{itemize}
        \item Man sollte auch den Absoluten Fehler untersuchen. War jedoch nicht Kern meiner Arbeit.
        \item Eine Mittelung führt Hier auf jedenfall auf eine Besserung egal welches (anderes beim relativen)
        \item Hier ist Intensität am besten und nicht Snesitivität (anders beim relativen)
        \item verschachtelung bringt auch hier eine Verbesserung (genauso beim relativen)
        \item Transformation sorgt bei allen drei schritten zu einer Verschlechterung (anders beim relativen Fehler)
        \item Bei der Verbesserung des relativen Fehlers wird sich auf den niedrigen Energiebereich konzentriert, da dort noch verbesserung nötig ist
              und kleine verbesserungen mehr einfluss haben, jedoch leidet meistens der bereich mit großen Energien was großen einfluss auf den Absoluten
              Fehler hat
        \item Beste Methode ist die untransformierte verschachtelte Schätzung
      \end{itemize}
    \item Folie 26:
      \begin{itemize}
        \item KEINE CUTS wie nur Bilder wo ganze Ellipse zu sehen ist
        \item Es muss auf jedenfall die Methode auf die Aufgabe angepasst werden.
        \item Für Spektren also relative Fehler: verschachtelte transformierte Methode  : Auflösung und Verzerrung können um 75\% gesenkt werden.
        \item Linienspektrum bei großen Energien: verschachtelte untransformierte Methode : Absoluter Fehler um 66\% senken
      \end{itemize}
    \item Folie 27:
      \begin{itemize}
        \item Abstand ist wichtig und führt dazu dassdie Hillas Parameter wichtiger werden, da dadurch diese ein Indiz auf die größe werden.
        \item Standen nur leider nicht funktionsfähig zur verfügung
        \item Die Transformation kann Aufgabenspezifische gewählt werden. Diese Tranfo gut für niedrige energien
        \item Für Linienspektren könnte man Gauß ähnliche Trafos nehmen die den Erwartungswert bei der gesuchten Energie hat und bijektiv sind.
        \item
      \end{itemize}
  \end{itemize}
\end{document}
