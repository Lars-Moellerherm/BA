\PassOptionsToPackage{unicode}{hyperref}
\documentclass[aspectratio=1610, professionalfonts, 9pt]{beamer}

\usefonttheme[onlymath]{serif}
\usetheme[showtotalframes]{tudo}


\usepackage{polyglossia}
\setmainlanguage{english}

% Mathematik
\usepackage{multicol}
\usepackage{graphicx}
\usepackage{amssymb}
\usepackage{amsmath}
\usepackage{xparse}
\usepackage{braket}
\usepackage{units}
\usepackage[locale=DE,separate-uncertainty=true,per-mode=reciprocal,output-decimal-marker={,},]{siunitx}
\usepackage[section]{placeins}
\usepackage{pdflscape}
\usepackage{expl3}
\usepackage{bookmark}
%Komma als Dezimaltrenner in der mathe Umgebung, um in Umgebungen wie [0, 2] ein Leerzeichen nach dem Komma zu erhalten einfach eins setzen
\usepackage{icomma}
\usepackage{cancel}
\usepackage{hyperref}
\usepackage{bookmark}
\usepackage{subfigure}

%%%%%%%%%%%%%%%%%%%%%%%%%%%%%%%%%%%%%%%%%%%%%%%%%%%%%%%%%%%%%%%%%%%%%%%%%%%%%%%%
%%%%%-------------Hier Titel/Autor/Grafik/Lehrstuhl eintragen--------------%%%%%
%%%%%%%%%%%%%%%%%%%%%%%%%%%%%%%%%%%%%%%%%%%%%%%%%%%%%%%%%%%%%%%%%%%%%%%%%%%%%%%%

%Titel:
\title{Energy Reconstruction by Random Forrest Regressor in sklearn for CTAs
real-time analysis}
%Autor
\author[L.~Möllerherm]{Lars Möllerherm}
%Lehrstuhl/Fakultät
\institute[Experimentelle Physik Vb]{Experimentelle Physik Vb \\  Fakultät Physik}

\begin{document}

\maketitle

\section{Data}
  \begin{frame}
    \frametitle{Steps done for a better Reconstruction}
    \begin{itemize}
      \item data: Monte Carlo data (gen_3) - pointlike and diffuse gamma (events).

      \item Used attributes for the RF:
      \begin{itemize}
        \item length
        \item width
        \item skewness
        \item curtosis
        \item intensity
        \item mean scaled width an length
      \end{itemize}

      \item steps:
      \begin{itemize}
        \item taking avarage over telescopes, which have seen the same event.
        \item taking an weighted mean(weights: intensity, telescope size, telescope sensitivity).
        \item taking the weighted(itensity) mean as a new attribute for another RF.
      \end{itemize}
    \end{itemize}

  \end{frame}


\section{Performance}
  \begin{frame}{Performance for different weights}
      %Graphics
  \end{frame}


  \begin{frame}{Performance for the nested modell}
    % Input: Complete ROC_Score for different weigths and Energy_binned ROC for
    % best over-all performing.
  \end{frame}

  \begin{frame}{Importanance of the Attributes}
    % Input: Complete ROC_Score for different weigths and Energy_binned ROC for
    % best over-all performing.
  \end{frame}

\section{Conclusion}
  \begin{frame}{Conclusion}
    % Suggest further study of weigthing by best weigth.
    % Possible problems: Bugs in MCs
  \end{frame}

\end{document}
