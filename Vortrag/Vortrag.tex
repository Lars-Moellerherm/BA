\PassOptionsToPackage{unicode}{hyperref}
\documentclass[aspectratio=1610, professionalfonts, 9pt]{beamer}

\usefonttheme[onlymath]{serif}
\usetheme[showtotalframes]{tudo}


\usepackage{polyglossia}
\setmainlanguage{english}

% Mathematik
\usepackage{multicol}
\usepackage{graphicx}
\usepackage{amssymb}
\usepackage{amsmath}
\usepackage{xparse}
\usepackage{braket}
\usepackage{units}
\usepackage[locale=DE,separate-uncertainty=true,per-mode=reciprocal,output-decimal-marker={,},]{siunitx}
\usepackage[section]{placeins}
\usepackage{pdflscape}
\usepackage{expl3}
\usepackage{bookmark}
%Komma als Dezimaltrenner in der mathe Umgebung, um in Umgebungen wie [0, 2] ein Leerzeichen nach dem Komma zu erhalten einfach eins setzen
\usepackage{icomma}
\usepackage{cancel}
\usepackage{hyperref}
\usepackage{bookmark}
\usepackage{subfigure}

%%%%%%%%%%%%%%%%%%%%%%%%%%%%%%%%%%%%%%%%%%%%%%%%%%%%%%%%%%%%%%%%%%%%%%%%%%%%%%%%
%%%%%-------------Hier Titel/Autor/Grafik/Lehrstuhl eintragen--------------%%%%%
%%%%%%%%%%%%%%%%%%%%%%%%%%%%%%%%%%%%%%%%%%%%%%%%%%%%%%%%%%%%%%%%%%%%%%%%%%%%%%%%

%Titel:
\title{Gamma-Hadron seperation by Random Forrest classification in sklearn for CTAs
real-time analysis}
%Autor
\author[R.~Dominik]{Rune Dominik}
%Lehrstuhl/Fakultät
\institute[Experimentelle Physik Vb]{Experimentelle Physik Vb \\  Fakultät Physik}

\begin{document}

\maketitle

\section{Data}
  \begin{frame}{Used MC data and attributes}
    \begin{itemize}
      \item Used MC data: CTA Prod 3b - pointlike gamma and proton (around 5.500.000 events).

      \item Used attributes:
      \begin{itemize}
        \item length
        \item width
        \item skewness
        \item curtosis
        \item intensity
        \item telescope type
      \end{itemize}

      \item Try different weights for multi-telescope prediction merging:
      \begin{itemize}
        \item Simple, arithmetic mean.
        \item Telescope based information: Focal length and mirror size.
        \item Array based information: Distance to core.
      \end{itemize}
    \end{itemize}
  \end{frame}


\section{Estimator performance}
  \begin{frame}{Estimator performance befor merging}
    \begin{itemize}
      \item Used estimator: scikit learn RandomForrestClassifier.
      \item Used hyperparameters: Out-of-the-box estimator with maximal depth cap
            around 15 (evaluated by GridSearch).
      \item Estimation by 5 time CV.
    \end{itemize}
    % Input: 2 grafics: Complete ROC_Score ofa_and_ofe and Energy_binned ROC_Score
    % ofa_and_ofe.
  \end{frame}


  \begin{frame}{Estimator performance after merging}
    % Input: Complete ROC_Score for different weigths and Energy_binned ROC for
    % best over-all performing.
  \end{frame}

  % Maybe nested modell.

\section{Conclusion}
  \begin{frame}{Conclusion}
    % Suggest further study of weigthing by best weigth.
    % Possible problems: Bugs in MCs
  \end{frame}

\end{document}
