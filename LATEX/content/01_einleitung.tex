\chapter{Einleitung}

Auf die Erdatmosphäre trifft eine große Menge von hochenergetischer kosmischer Strahlung.
Im Gegensatz zu der geladenen Teilchenstrahlung erreicht die Gammastrahlung auf direktem Weg die Erden.
Außerdem ist der Wirkungsquerschnitt im Gegensatz zu der Neutrinostrahlung groß genug, um eine ausreichende Statistik zu erlangen.
Daher wird versucht mithilfe der Gammastrahlung etwas über die unverstandenen Prozesse im Universum zu erfahren.
Die Energie der Strahlung liegt deutlich über der bisher erreichten Schwerpunktsenergie des LHC von $\approx\SI{13}{\tera\eV}$\cite{LHC} und weckt
besonders Interesse für die Entdeckung neuer Physik, die über das Standardmodell hinaus geht.
Als Beispiel dient die Suche nach Axionen, die in Blazaren und Gamma Ray Bursts enstehen oder die Suche nach Zerfällen der Dunklen Materie.

Da CTA jährlich $\SI{3.7}{\peta\byte}$\cite{Rohdaten} an Rohdaten verarbeiten muss, wird eine Computergesteuerte Datenanalyse
entwickelt und das Feld des maschniellen Lernens liefert die richtigen Werkzeuge für diese Art der Datenanalyse.
Aufgrund der Monte-Carlo Simulation, die das Luftschauer und die Reaktion des Teleskopes simuliert, kann ein Trainingsdatensatz erzeugt werden, auf dem
Methoden des überwachten Lernens trainiert werden können.
Mit diese trainierten Algorithmen der Untergrund vom Signal getrennt werden oder die Energie und Richtung rekonstruiert werden.

Das Interesse der CTA Organisation liegt bei extreme Objekte im Universum und deren Teilchenbeschleunigungsprozessen.
Dazu muss das Energiespektrum der Quelle gemessen werden und aufgrund der Erdgebundenheit kann das CTA die Energie der Gammastrahlung nur indirekt über
das atmosphärische Schauer messen.
Um nun die Ursprungsenergie des Photons zu erfahren muss die Energie rekonstruiert werden, dies wird mithilfe von Randomforest Regressoren durchgeführt.

Trotz der Tatsache, dass CTA das erste Cherenkov Teleskop Experiment, welches eine größere Anzahl an Teleskopen beinhaltet, wodurch bei einer Vielzahl von Events mehrere
Teleskope das gleiche Schauer sehen, wird noch für jedes Teleskop eine eigenständige Energie Rekonstruktion durchgeführt. Die einzelnen Ergebnisse variieren gleichverteilt
um den wahren Wert, daher sollte der Mittelwert deutlich näher an dem wahren Ergebnis liegen.
Da alle Teleskope einen unterschiedlichen Blickwinkel und Abstand zum Schauer besitzen, hat jedes Teleskop unterschiedlich viel Information und liegt somit
unterschiedlich nah am wahren Wert.

Eine Gewichtung der einzelnen Ergebnisse, würde diesen Informationsunterschied berücksichtigen.
Wenn nun eine erste Schätzung vorgenommen wurde, könnte ein zweiter Regressor eine feinere Vorhersage treffen, indem er sich auf Attribute konzentriert, die
für eine solche Unterscheidung wichtiger sind.

Wenn der Random Forest auf eine transformierte Energie trainiert wird, die über einen kleineren Energiebereich geht, könnte es dem
Algorithmus leichter Fallen den richtigen Wert zu schätzen, da eine geringere Streuung existiert.
