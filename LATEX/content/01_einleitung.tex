\chapter{Einleitung}

Auf die Erdatmosphäre trifft eine große Menge von hochenergetischer kosmischer Strahlung, welche uns viel über unverstandene
Prozesse im Universum verrät.
Die Gammastrahlung gelangt auf direktem Weg zur Erde, wodurch ihr Entstehungsort erfasst werden kann.
Die Energie der Strahlung liegt weit über der bisher erreichten Schwerpunktsenergie des LHC von ca. $\SI{13}{\tera\eV}$~\cite{LHC} und weckt
ein besonderes Interesse an der Entdeckung neuer Physik.
Als Beispiel dient die Suche nach Zerfällen der Dunklen Materie oder die Untersuchung der physikalischen Gesetze in extremer Umgebung.
Auch ein Großteil der Beschleunigungsprozesse von Teilchen im Universum sind noch nicht erklärt.
Um diese Fragen zu beantworten, muss die Gammastrahlung kosmischer Quellen genau untersucht werden.

Da CTA jährlich $\SI{3.7}{\peta\byte}$~\cite{Rohdaten} an Rohdaten verarbeiten muss, wird eine computergesteuerte Datenanalyse
genutzt und das Feld des maschinellen Lernens liefert die richtigen Werkzeuge.
Eine direkte Beobachtung mithilfe von Satelliten erlaubt es nicht genügend Ereignisse der hochenergetischen Strahlung zu beobachten.
Um eine ausreichende Fläche zu beobachten, wird eine indirekte, erdgebundene Beobachtung über das atmosphärische Schauer und das
entstehende Cherenkov-Licht gewählt.
Diese Lichtblitze werden mithilfe von Cherenkov-Teleskopen gemessen oder im Fall von CTA mit einer Gruppe von Teleskopen.

Um die Ursprungsenergie des Photons zu ermitteln, muss sie mithilfe von Random Forest Regressoren rekonstruiert werden.
Diese Random Forest Algorithmen gehören zum Bereich des überwachten maschinellen Lernens und können mithilfe von Trainingsdatensätzen,
die in Monte-Carlo-Simulationen entstehen, trainiert werden.

CTA besitzt mehr als $\num{100}$ Cherenkov-Teleskope, wobei mehrere Teleskope das selbe Schauer sehen können, jedoch wird momentan für jedes Teleskop eine eigenständige
Energierekonstruktion durchgeführt.
Da jedes Schauer nur ein Primärteilchen besitzt, müssen die Ergebnisse der einzelnen Teleskope zusammengefasst werden oder
eine Schätzung für jedes Schauerereignis und nicht für jedes Teleskop durchgeführt werden.
Da der Sensitivitätsbereich von CTA über vier Größenordnungen geht, könnte eine Transformation diesen Zielbereich verkleinern und
damit dem Random Forest die Schätzung erleichtern.
