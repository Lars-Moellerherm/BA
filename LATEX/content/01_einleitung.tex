\chapter{Einleitung}

Auf die Erdatmosphäre trifft eine große Menge von hochenergetischer kosmischer Strahlung und wechselwirkt mit den Luftmolekülen.
Bei dieser Strahlung wird unterschieden zwischen der Gammastrahlung, der geladenen Teilchenstrahlung und der Neutrinostrahlung.
Da die geladenen Teilchen durch die Magnetfelder auf ihrem Weg durch das Universum ständig die Richtung wechseln, bleibt ihr Ursprungsort
verborgen.
Die ungeladene Gamma- und die Neutrinostrahlung erreicht die Erde jedoch auf direktem Weg, wenn Gravitationslinsen Effekte
außer Acht gelassen werden. Die dünne Erdatmosphäre reicht jedoch nicht aus, um eine nennenswerte Neutrino Wechselwirkung zu beobachten.
Daher wird versucht mithilfe der Gammastrahlung etwas über die unverstandenen Prozesse im Universum zu erfahren.
Die Energie der Strahlung liegt deutlich über der bisher erreichten Schwerpunktsenergie des LHC von $\approx\SI{13}{\tera\eV}$\cite{LHC} und weckt
besonders Interesse für die Entdeckung neuer Physik, die über das Standardmodell hinaus geht.
Ein Beispiel dient die Suche nach Axionen, die in Blazaren und Gamma Ray Bursts enstehen oder die Suche nach Zerfällen der Dunklen Materie.

Da CTA jährlich $\SI{3.7}{\peta\byte}$\cite{Rohdaten} an Rohdaten verarbeiten muss, wird eine Computergesteuerte Datenanalyse
entwickelt und das Feld des maschniellen Lernens liefert die richtigen Werkzeuge für diese Art der Datenanalyse.
Aufgrund der Monte-Carlo Simulation, die das Luftschauer und die Reaktion des Teleskopes simuliert, kann ein Trainingsdatensatz erzeugt werden, auf dem
Methoden des überwachten Lernens trainiert werden können.
Und diese trainierten Algorithmen können auf die echten Daten angewendet werden, um das Event als Teilchenstrahlung oder
Gammaphoton einzuordnen oder um dessen Energie und Richtung zu rekonstruieren.

Das Interesse der CTA Organisation liegt bei extreme Objekte im Universum und deren Teilchenbeschleunigungsprozessen.
Dazu muss das Energiespektrum der Quelle gemessen werden und aufgrund der Erdgebundenheit kann das CTA die Energie der Gammastrahlung nur indirekt über
das atmosphärische Schauer messen.
Um nun die Ursprungsenergie des Photons zu erfahren muss die Energie rekonstruiert werden, dies wird mithilfe von Randomforest Regressoren durchgeführt.
Trotz der Tatsache, dass CTA das erste Cherenkov Teleskop Experiment, welches eine größere Anzahl an Teleskopen beinhaltet, wodurch bei einer Vielzahl von Events mehrere
Teleskope das gleiche Schauer sehen, wird noch für jedes Teleskop eine eigenständige Energie Rekonstruktion durchgeführt. Die einzelnen Ergebnisse variieren gleichverteilt
um den wahren Wert, daher sollte der Mittelwert deutlich näher an dem wahren Ergebnis liegen.
Da alle Teleskope einen unterschiedlichen Blickwinkel und Abstand zum Schauer besitzen, hat jedes Teleskop unterschiedlich viel Information und liegt somit
unterschiedlich nah am wahren Wert.
Eine Gewichtung der einzelnen Ergebnisse, würde diesen Informationsunterschied berücksichtigen.
Wenn nun eine erste Schätzung vorgenommen wurde, könnte ein zweiter Regressor eine feinere Vorhersage treffen, indem er sich auf Attribute konzentriert, die
für eine solche Unterscheidung wichtiger sind.

Die Sensitivität von CTA liegt in einem Energiebereich von $\SI{20}{\giga\eV}$ bis $\SI{300}{\tera\eV}$\cite{CTAscience}, jedoch kann der Randomforest
sehr niedrige Energien nicht schätzen, da es dort zu wenig Statistik gibt.
Durch eine linerare Transformation der Energien, kann für eine höhere Statistik im niegrig Energiebereich gesorgt werden.
Damit kann die Sensitivität der Teleskope im $\si{\giga\eV}$-Bereich genutzt werden.
