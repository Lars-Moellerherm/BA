\chapter{Zusammenfassung und Ausblick}
Die Energieauflösung des CTA Experiments ist noch nicht bei der Auflösung des MAGIC Experiments von $\num{0.2}$ für
$\SI{5000}{\giga\eV}$\cite{MAGIC_performance} in allen Energiebereichen angelangt, jedoch erzielen die Methoden dieser
Arbeit eine deutliche Verbesserung, obwohl keine Schnitte in den Daten vorgenommen wurden.

% Max Paper vorzitieren, um einen aktuellen Auflösungswert von MAGIC zu bekommen

\section{Fazit der Arbeit}

Durch das Bilden des arithmetischen Mittels über aller Teleskope ergibt sich kein Performancegewinn, da die Schätzungen
eine Verzerrung besitzen, die durch eine Mittelwertbildung nicht verbessert werden kann.
Die Bildung des Median hingegen ermöglicht eine kleine Verbesserung, da einzelne Schätzungen, die möglicherweise
eine falsche Schätzung aufgrund der geringen Sicht des Teleskopes auf das Schauer abgeben, herausgefiltert werden.
Eine geeignete Gewichtung bei der Mittelwertbildung, die diesen Effekt beim arithmetischen Mittel erreichen könnte, wurde
nicht gefunden, da die Fehlschätzungen über einen Bereich von fünf Größenordnungen gehen können.
Eine Schätzung mithilfe eines zweiten Random Forests, der eventspezifische Informationen und eine
Schätzung des ersten Algorithmus bekommt, liefert einen enormen Performancegewinn, da durch die erste Schätzung eine
Trennung vorgenommen werden kann, die eine anschließende energiespezifischere Analyse ermöglicht.
Das Problem des großen Zielbereichs wird mithilfe einer logarithmischen Transformation gelöst, wodurch ein weitere
Verbesserung der Auflösung und der Verzerrung verzeichnet wird.
Eine verschachtelte Analyse mit einem transformierten Zielbereich, die jedoch einen nicht parallelisierbaren Zeitaufwand bedeutet,
führt die Energieschätzung des CTA-Experiments mit einer ungefähren Auflösung von $0.4$ in weiten Teilen des
Energiebereichs nahe an die Performance des MAGIC Projekts heran.

\section{Perspektiven}

Eine deutliche Verbesserung wird die Hinzunahme des Abstandes von Teleskope und Schauer und die Schätzung des ersten
Wecheselwirkungspunktes des primär Teilchens mit der Atmosphäre als zusätzliche Attribute bewirken, da diese wichtige
charakteristische Merkmale für die Energieschätzung sind. Außerdem dürften die Hillas Parameter wichtiger werden, da durch
die Distanz zum Schauer die Hillasparameter ein Indiz auf die größe des Schauers geben können.
Dadurch wird die Performance der einfachen Schätzung näher an die des verschachtelten Waldes herankommen.
Diese Attribute stehen zum Zeitpunkt der Arbeit jedoch nicht funktionsfähig zur Verfügung.

Zusätzlich wäre es von Interesse den Abstand als Gewicht zu testen, da dieser ein direktes Indiz auf die Sichtbarkeit
liefert.
Vielleicht liefern andere Gewichte, Kombinationen von Gewichten oder skalierte Gewichte einen Performancegewinn.

Von Interesse wäre es andere Transformationen zu testen, die den Zielbereich weiter verkleinern und für eine gleichverteilte
Statistik in allen Energiebereichen sorgen.
Vielleicht können auch unterschiedliche Transformationen angewendet werden, jenachdem welcher Energiebereich für die
Beobachtung von Interesse ist und somit die beste Performance für jede Analyse individuell herausgeholt werden kann.

Ein großer Grund für den Erfolg des verschachtelten Modells und der Transformationsmethode ist die Tatsache, dass für das
Ausbauen der Entscheidungbäume ein Kriterium genutzt wir, welches den absoluten Fehler minimiert, wobei das Anforderungsprofil des
CTA eine Minimierung des relativen Fehlers verlangt.
Das Entwickeln von anderen Kriterien oder das Nutzen von anderen Regressionsmethoden, die die Minimierung des relativen Fehlers
anstreben, würden die Analyse anforderungsorientiert verbessern.
