\thispagestyle{plain}

\section*{Kurzfassung}
In dieser Arbeit wird die Energie Rekonstruktionsanalyse von hochenergetischen kosmischen Photonen, welche durch das Cherenkov Telescope Array beobachtet
werden, optimiert.
Bis zu diesem Zeitpunkt wird für jedes Teleskop eine eigenständige Energieschätzung durch Algorithmen des überwachten Maschinellen Lernens durchgeführt.
Da beim CTA die Möglichkeit besteht, dass mehrere Teleskope ein Event sehen, wird durch Zusammenfassen der einzelnen Ergebnisse die
Energieschätzung verbessert.
Eine einfache oder gewichtete Mittelung über die Schätzungen liefert keinen Performancegewinn, eine verschachtelte und eventspezifische Regression jedoch
führt auf eine Verringerung des relativen Fehlers und auf eine Verbesserung der Energieauflösung.
Der große Zielbereich der Analyse führt auf einen starken Performanceverlust bei niedrigen Energien, was durch eine geeignete bijektive Transformation
geändert wird, wodurch die Schätzung in diesem Energiebereich verbessert werden kann.
Die getesteten Methoden führen auf eine Viertelung des relativen Fehlers und der Energieauflösung in großen Teilen des Sensitivitätsspektrums des CTA.

\section*{Abstract}
\begin{english}
This thesis is optimizing the energy reconstruction for heigh energy cosmic photons, which are detected by the Cherenkov Telescope Array.
For now there is a prediction by supervised machine learning regressors for every single telescope.
By the fact, that CTA is an array of telescopes, which can measure the same event, the analyse can be optimized by summarizing the results of these telescopes.
The simple or the weighted average over each prediction does not bring Performancegain, but an nested model, which predicts for every event, causes an improvement
for the relativ error and för the energyresolution of CTA.
Because of the wide spread image of the estimator there is a big Performanceloss for low energies.
This can be managed by an Transformation of the output, which ensures an improvment for the Performance.
These methods are improving the bias of the realtiv error and the energy resolution by an factor of four in a big energy range of CTA.
\end{english}
