\thispagestyle{plain}

\section*{Kurzfassung}
In dieser Arbeit wird die Energie Rekonstruktionsanalyse für hochenergetischen kosmischen Photonen, welche durch das Cherenkov Telescope Array beobachtet
werden, optimiert.
Bis zu diesem Zeitpunkt wird für jedes Teleskop eine eigenständige Energieschätzung durch Algorithmen des überwachten Maschniellen Lernens durchgeführt.
Da beim CTA die Möglichkeit besteht, dass mehrere Teleskope ein Event sehen, wird durch einfaches Mitteln über diese eigenständigen Ergebnisse die
Energieschätzung deutlich besser.
Eine Gewichtung des Mittelwerts mit der gemessenen Cherenkov Photonen Anzahl für jedes Teleskops führt auf eine weitere Verbesserung und wenn die erste
Schätzung als neues Attribut in einer weiteren Regressionsanalyse verwendet wird, kann die tatsächliche Energie noch genauer vorhergesagt werden.
Aufgrund der geringen Sensitivität der Teleskope bei niedrigen Energien fehlt es dem Regressor an Statistik, um in diesem Bereich eine genaue Vorhersage
treffen zu können.
Um diese Statstik zu erhöhen, wird eine Transformation der Energieverteilung vorgenommen, was zwar zu einer höheren Genauigkeit bei geringen Energien führt,
jedoch zu großen Ungenauigkeiten in den anderen Energiebereichen.

\section*{Abstract}
\begin{english}
This thesis is optimizing the energy reconstruction for heigh energy cosmic photons, which are detected by the Cherenkov Telescope Array.
For now there is a prediction by supervised machine learning regressors for every single telescope.
By the fact, that CTA is an array and more then one telescope can measure the same event, the analyse can be optimized by averaging over these telescopes.
Another improvement can be achieved by using the Chrenkov photon number from every telescope as an weight for the average.
If these first predictions are used for a second regressor, the predictions are getting more accurate.
The telescopes are less sensitive for low energetic cosmic rays and so there are less observed events and the estimations are getting vague.
By using the right transformation for the energydistribution, the algorithm can better predict low energy events but getting worse for the higher energies.
\end{english}
