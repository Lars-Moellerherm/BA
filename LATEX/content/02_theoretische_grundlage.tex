\chapter{Theoretische Grundlagen}

\section{Gammaastronomie}

Im Universum gibt es zahlreiche Prozesse, bei denen hochenergetische Teilchen entstehen, oder auf diese Energien beschleunigt werden.
Bei diesen Teilchen handelt es sich zum großen Teil um Protonen oder leichten  Atomkernen bis hin zu Eisen, aber auch Elektronen, Myonen
sind Bestandteil der Kosmischen Strahlung.
Diese Teilchen können durch Druckwellen, wie sie bei Sternexplosionen vorkommen, beschleunigt werden. Dieser Prozess wird durch das
Modell der Fermi-Beschleunigung erster und zweiter Ordnung beschrieben. Bei der Fermi-Beschleunigung wird das Medium, indem die Druckwelle
propagiert, durch ein Plasma beschrieben, welches Magnetfeld Störungen mit sich führt. Ein geladenes Teilchen trifft nun auf eine
solche Störung, welche sich mit einer Geschwindigkeit $v$ durch das Medium bewegt. Durch das Magnetfeld und die Lorentzkraft wird
das Teilchen nun mit einem Winkel $\theta$ gestreut. Durch diesen elastischen Stoß wird das Teilchen beschleunigt. Wenn nun alle
Möglichen Winkel berücksichtigt werden, ergibt sich eine Energiegewinn von
\begin{equation*}
  \left\langle \frac{\delta E}{E} \right\rangle = \frac{8}{2}\left(\frac{v}{c}\right)^2\text{ .}
\end{equation*}
Bei einer typischen Druckwellengeschwindigkeit von $v=\SI{e4}{\m\per\s}$\cite[14]{HESS} ergibt sich ein Energiegewinn von
$\SI{4.5e-9}{\m\per\s}$. Dies ist die Fermibeschleunigung zweiter Art und kann die große Beschleunigung in Supernovae nicht alleine erklären.
Ein größeren Beitrag liefert die Fermibeschleunigung erster Art, bei der die Teilchen durch mehrfaches Durchqueren der Schockfront
beschleunigt werden. Der Energiegewinn beträgt für alle Streuwinkel
\begin{equation*}
  \left\langle \frac{\delta E}{E} \right\rangle \approx \frac{2}{3}\frac{\delta v}{c} \text{ ,}
\end{equation*}
wobei $\delta v$ der Geschwindigkeitsunterschied zwischen dem Materie hinter und vor der Schockwelle ist.

Jedoch erklären diese Prozesse nicht die ultra hochenergetischen Teilchen. Die Phänomene die Materie bis auf diese Energien beschleunigen,
sind noch nicht vollständig erforscht, jedoch sind Jets, die bei stark akkretierender Materie entstehen, vielversprechende Kandidaten.
Bei stark akkretiernder Materie, die auf ein Objekt fällt, strömt ein Großteil der Materie senktrecht zur Akkretionscheibe, dem Jet,
ab, damit der Drehimpuls erhalten bleibt.
Ein weiterer möglicher Prozess ist der Zerfall von Dunkler Materie oder im speziellen von Axionen der vermutlich hochenergetische
Teilchen erzeugt.

Wichtige Prozesse bei der Erzeugung hochenergetischer Photonen sind die Wechselwirkungen von Photonen und Elektronen. Die
Elektron-Photon Wechelwirkung wird beschrieben durch die
Compton-Streuung, wobei ein Photon mit einem Elektron elastisch stößt, durch die Paarerzeugung, wobei aus einem Photon ein Elektron
und ein Positron enstehen, durch die Annihilation, welcher der Umkehrprozess zur Paarerzeugung ist und aus einem Elekron Positron Paar
ein Photon entsteht, und der Bremsstrahlung, welche bei der Impulsänderung gelandener Teilchen entsteht.

Diese hochenergetischen Photonen oder auch Gammastrahlung genannt, können entweder direkt mithilfe von Satelliten im Weltraum oder
indirekt über das in der Atmosphäre enstehende Schauer mithilfe von Cherenkov Teleskopen auf der Erdoberfläche beobachtet werden.
Die Gammastrahlung ist so energiereich, dass die Satelliten nur mithilfe von Szinillationszählern diese messen können und nicht
durch Spiegelteleskope. Diese energiereiche Strahlung sorgt jedoch dafür, dass in der Erdatmosphäre durch die Wechselwirkung mit
den Luftmolekülen ein hochrelativistisches Teilchenschauer ensteht. In diesem Schauer entstehen geladene Teilchen, die eine Geschwindigkeit
besitzen, die über der Lichtgeschwindigkeit in Luft liegt. Geladene Teilchen polarisieren die Luftmoleküle für kurze Zeit, wodurch sie
elektromagnetische Wellen aussenden. Wenn sich nun das geladene Teilchen schneller als die Lichtgeschwindigkeit des Mediums bewegt,
können sich die Wellen nicht mehr destruktiv überlagern und es ensteht Cherenkov-Licht, welches sich kegelförmig ausbreitet und von
den Teleskopen am Boden beobachtet werden kann. Das Spektrum der Cherenkov-Strahlung ist kontinuierlich, die Intensität pro Frequenz ist
jedoch proportional zur Frequenz im sichtbaren Bereich und daher wird das Cherenkov-Licht als bläulich wahrgenommen.

\section{Cherenkov Teleskope Array (CTA)}

Das Cherenkov Teleskop Array ist ein Zukunftsprojekt eines internationalen Zusammenschlusses von Instituten und Universitäten aus
11 Ländern. Es sollen 89 Teleskope auf der Südhalbkugel in Chile gebaut werde und 19 Teleskope auf der Nordhalbkugel auf La Palma.
Es gibt drei verschiedene Größen an Teleskopen, welche unterschiedliche Sensitivitäten haben. Es gibt das LST (Large Sized Telescope)
welches eine Spiegelgröße von $\SI{23}{\m}$ besitzt, das MST (Medium Sized Telescope) mit einer Größe von $\SI{11.5}{\m}$ oder $\SI{9.7}{\m}$
und das SST (Small-Sized Telescope) welches $\SI{4.3}{\m}$ oder $\SI{4.0}{\m}$ groß ist.
Das LST besitzt eine volle Sensitivität in einem Energiebereich von $(20-150)\,\si{\giga\eV}$, das MST in einem Bereich von
$\SI{150}{\giga\eV}-\SI{5}{\tera\eV}$ und das SST in einem Bereich von $(5-300)\,\si{\tera\eV}$.

\section{Energie Rekonstruktion}
%Hillas Parameter

\section{Maschinelles Lernen}

\section{Random Forest Regressor}

\section{Mean Scaled Value}


\section{Transformation von Wahrscheinlichkeitsverteilungen}
