\chapter{Theoretische Grundlagen}

\section{Gammaastronomie}

Im Universum gibt es zahlreiche Prozesse, bei denen hochenergetische Teilchen entstehen, oder auf diese Energien beschleunigt werden.
Bei diesen Teilchen handelt es sich zum großen Teil um Protonen oder leichten  Atomkernen bis hin zu Eisen, aber auch Elektronen, Myonen
sind Bestandteil der Kosmischen Strahlung.
Diese Teilchen können durch Druckwellen, wie sie bei Sternexplosionen vorkommen, beschleunigt werden.
Dieser Prozess wird durch das Modell der Fermi-Beschleunigung erster und zweiter Ordnung beschrieben.
Bei der Fermi-Beschleunigung wird das Medium, indem die Druckwelle propagiert, durch ein Plasma beschrieben, welches Magnetfeldstörungen mit sich führt.
Wenn ein geladenes Teilchen auf eine  solche Störung, welche sich mit einer Geschwindigkeit $v$ durch das Medium bewegt, trifft, wird es durch das Magnetfeld
mit einem Winkel $\theta$ gestreut.
Durch diesen elastischen Stoß wird das Teilchen beschleunigt.
Wenn nun alle möglichen Winkel berücksichtigt werden, ergibt sich eine Energiegewinn von
\begin{equation*}
  \left\langle \frac{\delta E}{E} \right\rangle = \frac{8}{2}\left(\frac{v}{c}\right)^2\text{ .}
\end{equation*}
Bei einer typischen Druckwellengeschwindigkeit von $v=\SI{e4}{\m\per\s}$\cite[14]{HESS} ergibt sich ein Energiegewinn von $\SI{4.5e-9}{\m\per\s}$.
Dies ist die Fermibeschleunigung zweiter Art und kann die große Beschleunigung in Supernovae nicht alleine erklären.
Ein größeren Beitrag liefert die Fermibeschleunigung erster Art, bei der die Teilchen durch mehrfaches durchqueren der Schockfront beschleunigt werden.
Der Energiegewinn beträgt für alle Streuwinkel
\begin{equation*}
  \left\langle \frac{\delta E}{E} \right\rangle \approx \frac{2}{3}\frac{\delta v}{c} \text{ ,}
\end{equation*}
wobei $\delta v$ der Geschwindigkeitsunterschied zwischen dem Materie hinter und vor der Schockwelle ist.

Jedoch erklären diese Prozesse nicht die ultra hochenergetischen Teilchen.
Die Phänomene die Materie bis auf diese Energien beschleunigen, sind noch nicht vollständig erforscht, es wird zum Beispiel nach möglichen elektrischen
Potentialunterschieden gesucht, die groß genaug sind um die Teilchen zu beschleunigen.
Ein weiterer möglicher Prozess ist der Zerfall von Dunkler Materie oder im speziellen von Axionen, der vermutlich hochenergetische Teilchen erzeugt.

Diese hochenergetische Teilchenstrahlung erzeugt durch Wechselwirkungen oder Zerfälle Gammastrahlung.
Wichtige Prozesse bei der Erzeugung hochenergetischer Photonen sind die Wechselwirkungen von Photonen und Elektronen.
Die Elektron-Photon Wechelwirkung wird beschrieben durch die
Compton-Streuung, wobei ein Photon mit einem Elektron elastisch stößt, durch die inverse Compton_Streuung, wobei das geladene Teilchen Energie auf das Photon
überträgt  durch die Paarerzeugung, wobei aus einem Photon ein Elektron
und ein Positron enstehen, durch die Annihilation, welcher der Umkehrprozess zur Paarerzeugung ist und aus einem Elekron Positron Paar
ein Photon entsteht, und durch die Bremsstrahlung, welche bei der Impulsänderung gelandener Teilchen entsteht.

Diese hochenergetischen Photonen oder auch Gammastrahlung genannt, können entweder direkt mithilfe von Satelliten im Weltraum oder indirekt über das in
der Atmosphäre enstehende Schauer mithilfe von Cherenkov Teleskopen auf der Erdoberfläche beobachtet werden.
Die Gammastrahlung ist so energiereich, dass die Satelliten nur mithilfe von Szinillationszählern diese messen können und nicht durch Spiegelteleskope.
Diese energiereiche Strahlung sorgt jedoch dafür, dass in der Erdatmosphäre durch die Wechselwirkung mit den Luftmolekülen ein hochrelativistisches
Teilchenschauer ensteht.
In diesem Schauer entstehen geladene Teilchen, die eine Geschwindigkeit besitzen, die über der Lichtgeschwindigkeit in Luft liegt.
Geladene Teilchen polarisieren die Luftmoleküle für kurze Zeit, wodurch sie elektromagnetische Wellen aussenden.
Wenn sich nun das geladene Teilchen schneller als die Lichtgeschwindigkeit des Mediums bewegt, können sich die Wellen nicht mehr destruktiv
überlagern und es ensteht Cherenkov-Licht, welches sich kegelförmig ausbreitet und von den Teleskopen am Boden beobachtet werden kann.
Das Spektrum der Cherenkov-Strahlung ist kontinuierlich, die Intensität pro Frequenz ist jedoch proportional zur Frequenz im sichtbaren Bereich und
daher wird das Cherenkov-Licht als bläulich wahrgenommen.

\section{Cherenkov Teleskope Array (CTA)}

Das Cherenkov Teleskop Array ist ein Zukunftsprojekt eines internationalen Kollaboration von 210 Instituten aus 32 Ländern\cite{CTA_consortium}.
CTA ist der nächste Schritt in der hochenergie Gammastronomie.
Mit einer Gesamtanzahl von 108 Teleskopen hat das Array nach Simulationen zur Folge in seinem Hauptenergiebereich eine Sensitivität von $\SI{0.1}{\percent}$
des Energieflusses von dem Krebsnebels, was ungefähr zehn mal sensitiver als das HESS-Experiment ist\cite{CTA_paper}.
Da der Teilchenfluss $F$ der kosmischen Strahlung in dem Sensitivitätsbereich von CTA dem Potenzgesetz $F \propto E^{-2.7}$\cite[5]{Cosmic_rays} folgt,
treffen bei einer Energie $E$ von $\SI{1}{\tera\eV}$ noch $\SI{1}{\per\m\squared\per\s}$ auf die Erdatmosphäre.
Um dennoch genug Statistik zu haben, muss eine möglichst große Himmelfläche observiert werden.
CTA kann durch die große Anzahl an Teleskopen bei einer Beobachtungszeit von $\SI{0.5}{\hour}$ Photonen mit einer Energie von $\SI{1}{\tera\eV}$ auf einer
Fläche von $\SI{10e6}{\m\squared}$\cite{CTA_ob} beobachten.
Durch 3 verschiedene Größen von Teleskopen, das LST (Large Sized Telescope) mit einer Spiegelgröße von $\SI{23}{\m}$, das MST (Medium Sized Telescope)
mit einer Größe von $\SI{11.5}{\m}$ oder $\SI{9.7}{\m}$ und das SST (Small-Sized Telescope), welches eine Größe von $\SI{4.3}{\m}$ oder $\SI{4.0}{\m}$
besitzt, kann CTA Photonen mit Energien von $\SI{30}{\giga\eV}$ bis $\SI{300}{\tera\eV}$ detektieren.
Dieses breite Spektrum ist wichtig, um die ganze Bandbreite der Beschleunigungsprozesse im Universum untersuchen zu können.
Insbesondere kann das "knee" des Energiespektrums bei $\SI{3e15}{\eV}$ genau untersucht werden.
Die hohe Sensitivität und die niedrige Energieuntergrenze ermöglichen die Entdeckung neuer Quellen mit einer starken Rotverschiebung, die nur bei niedrigen
Energien sichtbar sind, da die höher energetische Gammastrahlung mit der Hintergundstrahlung wechselwirkt.

\section{Energie Rekonstruktion}
%Hillas Parameter

\section{Maschinelles Lernen}

\section{Random Forest Regressor}

\section{Mean Scaled Value}


\section{Transformation von Wahrscheinlichkeitsverteilungen}
